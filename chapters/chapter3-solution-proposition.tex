% ----------------------------------------------------------
% Capítulo 3: Solução proposta
% ----------------------------------------------------------

\section{Sistema proposto}

Com o objetivo de mitigar o problema proposto anteriormente, este trabalho se propõe a desenvolver 
um dispositivo embarcado com a capacidade de ler as taxas de poluentes no ar das regiões onde os 
ciclistas trafegam. Esses dados serão coletados pelo servidor e transformados em informação, tornando 
possível a visualização da mesma, bem como localizar pontos de concentração de má qualidade do ar. 
O sistema também será capaz de indicar se o ciclista está inalando mais ar poluído do que um usuário 
de transportes públicos ou privados e se o dano causado supera os benefícios para a saúde 
consequentes do exercício, além de sugerir rotas que desviem das regiões de picos de poluição.

O dispositivo embarcado na bicicleta visa ser portátil, de baixo consumo de energia, alta 
disponibilidade e persistir os dados no servidor. Para isso, ele se comunicará com a internet 
através do LoRaWAN (do inglês: ``Low Power Wide Area Network''), que consiste, como exposto nos 
capítulos anteriores, em uma rede de grande alcance e baixo consumo de energia.

\section{Metodologia}

Esta pesquisa utilizará o método dedutivo, desenvolvendo cadeias de raciocínio com premissas bem 
definidas para chegar a uma conclusão clara e objetiva. Além disso, este trabalho consiste em uma 
pesquisa quantitativa descritiva, ou seja, visa resolver um problema prático específico; analisar os 
dados, suas relações e impactos de forma concisa; e envolver estratégias de coleta de dados, neste 
caso, através de sensores, por amostragem não probabilística intencional, onde o grupo estudado será 
formado por ciclistas.

\section{Premissas}

1. A pesquisa será baseada em dados obtidos no estado do Rio de Janeiro.
2. Será usada a rede LoRa.
3. Um protótipo será feito utilizando microcontroladores e sensores de baixo consumo de energia.
4. O grupo que será estudado na pesquisa será formado por ciclistas.
5. O dispositivo embarcado desenvolvido precisa ser portátil para acoplar na bicicleta.
6. Deverá ser desenvolvido uma ferramenta de visualização das informações.
7. O projeto deverá considerar a localização do usuário.