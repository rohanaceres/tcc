% ----------------------------------------------------------
% Introdução
% ----------------------------------------------------------

Em 2015, na sede da ONU de Nova Iorque, aconteceu um encontro entre todos 
os países das Nações Unidas, com o objetivo de traçar metas visando o 
desenvolvimento sustentável. Com isso, uma agenda foi desenvolvida e nomeada 
de Objetivos de Desenvolvimento Sustentável. 

Dentre os 17 objetivos identificados, o 11º é descrito como \"Tornar as cidades 
e os assentamentos humanos inclusivos, seguros, resilientes e sustentáveis\", 
chamando a atenção para as taxas alarmantes de emissão de gases residuais 
principalmente em áreas urbanas. Em 2015 foi registrado que metade da população 
mundial vive em grandes cidades e a previsão para 2030 é que essa porcentagem 
suba para 60\%, ou seja, as cidades que ocupam aproximadamente apenas 2\% da área 
do planeta irão abrigar 60\% da humanidade, consumindo 80\% da energia produzida 
e causando 75\% da emissão de gases poluentes na atmosfera. 

No Brasil, o crescimento desgovernado nas metrópoles tem ameaçado a infraestrutura 
não planejada, enfatizando problemas como oferta de água potável, esgoto, saúde 
pública, transporte, qualidade do ar, conservação de energia, diminuição do impacto 
gerado pelo trânsito, dentre outros. Em 2016, segundo uma pesquisa implementada 
pela Organização Mundial da Saúde (OMS), 92\% da população mundial esteve exposta a 
níveis alarmantes de poluição e 7  milhões de pessoas morreram devido à degradação 
ambiental - sendo 4 milhões relacionadas ao uso da madeira, carvão e biomassa, e 3 
milhões aos gases residuais liberados por veículos automotores. Surpreendentemente, 
esse número excede a quantidade de mortes por AIDS (Síndrome da imunodeficiência 
adquirida) e malária juntos.

De acordo com a OMS, os problemas relacionados à exposição constante de poluentes são 
o AVC (Acidente Vascular Cerebral), problemas respiratórios, diabetes, doenças 
cardiovasculares, câncer e infertilidade. Tais problemas podem ser influenciados ou 
catalisados por compostos orgânicos voláteis, como tinta de parede, revestimento de 
carpete e produtos de limpeza, até os gases liberados pelas industrias e veículos 
automotores.

Quando gases ou partículas emitidos pela ação humana atingem concentrações 
suficientemente altas que causam danos diretos à população, seja ela humana ou não, 
um requisito básico para o bem estar de todos é negligenciado.

O projeto estadual MonitorAR Rio coleta dados de emissão de gases poluentes desde 2010, 
nas regiões Centro, Copacabana, São Cristóvão, Tijuca, Irajá, Bangu, Campo Grande, 
Pedra de Guaratiba e Recreio. Em 2011 e 2012, os seguintes dados referentes a taxa de 
gases poluentes coletados foram:

% TODO: ADICIONAR GRAFICOS.

Dentro dessa perspectiva, voltados para a questão da qualidade do ar e do impacto 
gerado pelo trânsito, urbanistas de todo o mundo se reuniram para discutir e desenvolver 
programas urbanísticos de baixo nível de agressão ambiental, bem como buscar definir um 
desenvolvimento socioeconômico que melhore e não destrua o meio ambiente natural e 
construído.

Algumas cidades europeias, percebendo a importância de transportes alternativos, 
priorizaram a bicicleta no trânsito, com a intenção de diminuir a poluição ambiental, 
humanizar as ruas e reduzir a quantidade de acidentes. Para isso o governo disponibilizou 
bicicletas públicas e investiu na construção de redes cicloviárias interligadas a outros 
transportes públicos, como metrô, barcas, trens e etc. No Brasil, um exemplo de programa 
semelhante é o BikeRio. Esse programa foi desenvolvido pelo governo do Rio de Janeiro em 
parceria com o Itaú, promovendo o aluguel de bicicletas por um valor mensal ou diário 
acessível. Ainda que o BikeRio seja um sucesso, o uso da bicicleta ainda encontra fortes 
obstáculos, especialmente por causa da falta de cidadania e respeito no trânsito e pelas 
ciclovias escassas e de má qualidade - mesmo em cidades como Curitiba e Rio de Janeiro, 
onde existe um investimento mais forte nesse contexto.

Além de usada como meio de transporte, a bicicleta é boa para a saúde, sendo considerada 
uma das melhores técnicas para prevenir e tratar a hipertensão, o infarto do miocárdio e 
o colesterol alto.

Este trabalho tem como objetivo mensurar a quantidade de ar poluído inalado, em média, pelo 
ciclista carioca das regiões urbanas. Um dispositivo embarcado será instalado na bicicleta, 
com a capacidade de captar o nível de gases poluentes durante o trajeto do ciclista. Os 
dados poderão ser consultados em tempo real através de uma página web.

Para alcançar este propósito, será criado um protótipo com o Arduino Trinket, principalmente 
por ser uma placa pequena, de fácil implementação e com diversas portas I/O, flexibilizando 
a inclusão dos sensores. O protótipo será planejado baseado em redes LoRa. Estas são redes 
sem fio de alto alcance e baixo consumo de energia, criadas especificamente para promover a 
comunicação entre dispositivos embarcados (IoT). Com isso, é possível acessar os dados 
captados por cada usuário em tempo real.

No capítulo 2, \"Resumo histórico e estado da arte\", serão apresentadas diversas pesquisas 
referentes ao tema do trabalho, evidenciando as tecnologias e soluções encontradas até o 
presente momento, para mitigar os problemas relacionados; no capítulo 3 - \"Proposição do 
problema\" - o problema, encontrado através das pesquisas, será brevemente descrito; e no 
capítulo 4 - \"Solução proposta\" - a solução proposta será apresentada, abordando os 
aspectos tecnológicos e suas premissas, contendo uma visão geral do sistema a ser 
desenvolvido.