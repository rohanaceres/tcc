% ----------------------------------------------------------
% Capítulo 2: Proposição do problema
% ----------------------------------------------------------

Atualmente, estamos sofrendo as consequências da revolução industrial. Como retratado anteriormente, 
uma dessas consequências é o grande crescimento das áreas urbanas e a má qualidade do ar nessas 
regiões, causando problemas de saúde à população. Visando resolver esse problema, estudos 
urbanísticos concluíram que aumentar o uso das bicicletas nas cidades pode mitigar os fatores 
prejudiciais e até se tornar um benefício para os ciclistas. 

Porém, como abordado nos capítulos anteriores, o ciclista, por estar praticando um exercício físico, 
inala o ar com mais frequência. Portanto, a quantidade de ar tóxico ventilando pelos pulmões é maior, 
visto que geralmente o ciclista passa menos tempo no trânsito do que o consumidor dos transportes 
públicos ou particulares.

Tornar o uso de bicicletas a principal ferramenta de mobilidade urbana pode causar, graças à 
poluição, problemas de saúde a longo prazo. Nesta perspectiva, este trabalho visa mensurar os fatores 
prejudiciais à saúde, bem como propor uma solução tecnológica para o problema,  que serão 
apresentados nos capítulos posteriores. 
